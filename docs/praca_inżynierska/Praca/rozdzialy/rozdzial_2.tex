\chapter{Przegląd rozwiązań}

Ten rozdział poświęcony zostanie przeglądowi istniejących rozwiązań pokrywających się funkcjonalnościami z aplikacją Concerto, ze szczególnym uwzględnieniem rozwiązań, o które system został rozbudowany w ramach realizacji tej pracy. Przegląd ten obejmie systemy dostarczające repozytorium zasobów (skoncentrowane na przechowywaniu zasobów muzycznych), streamingi wideo, systemy zarządzania treścią oraz dostawców lokazliacji do aplikacji webowych. Dla każdej z kategorii przedstawione zostaną teoretyczne aspekty i wymagania stawiane przed rozwiązaniem, a następnie przykłady istniejących narzędzi wraz z analizą podejść do ich zaadresowania.

\section{Repozytoria zasobów muzycznych}

Cyfrowe repozytoria zasobów muzycznych przeznaczone do nauki muszą spełniać szereg wymagań, aby zapewnić użytkownik akceptowalny poziom usługi. Wymagania te obejmują zarówno aspekty użytkowe, jak i techniczne. Systemy muszą zagwarantować bezpieczeństwo środowiska, w którym przechowywane są zasoby, aby użytkownicy nie musieli martwić się o bezpieczeństwo swoich danych, zarówno w kontekście zabezpieczeń systemu, jak i zapewnienia ich trwałości w repozytorium. Profesjonalne rozwiązania powinny umożliwać łatwy dostęp do treści poprzez intuicyjne interfejsy lub wyszukiwarkę plików oraz umożliwić kategoryzowanie materiałów, jak i ich filtrowanie oraz sortowanie. Aspekty te są szczególnie ważne dla systemów przechowujących duże ilości zasobów, ponieważ niezależnie od wielkości repozytorium, uczniowie jak i nauczyciele powinni mieć możliwość szybkiego dostępu do interesujących ich zasobów. Kolejnym ważnym aspektem jest wsparcie szerokiego zakresu plików muzycznych, takich jak notatki nutowe, audio, wideo, pliki MIDI lub projekty w formacie odpowiadającym popularnym programom do tworzenia muzyki. Każda z wymienionych kategorii zasobów może być reprezentowana przez różne formaty plików (dla przykładu audio może być w formatach takich jak MP3, WAV, M4A), a więc systemy powinny być w stanie obsługiwać wszystkie popularne formaty plików, jednocześnie uwzględniając idące za tym ograniczenia związane z bezpieczeństwem aplikacji. Dodatkowo systemy powinny umożliwiać transfer sieciowy dużych plików, gdyż zasobami mogą być między innymi godzinne nagrania z koncertów, czy ścieżki dźwiękowe do filmów. Autoryzacja powinna odbywać się na poziomie pojedyńczych plików z możliwością ich dziedziczenia, aby zapewnić wysoki poziom ochrony danych we współdzielonym repozytorium, jak i wygodę nadawania uprawnień. W aplikacjach omawianego typu często spotyka się połączenie modeli autoryzacji poprzez role (RBAC) z modelem autoryzacji poprzez konto (ABAC), aby odwzierciedlić ograniczenia wynikające z wymagań biznesowych, takich jak przynależność do określonej organizacji lub posiadanie konkretnego planu licencyjnego. Repozytoria powinny być skalowalne i zapewniać wysoką wydajność niezależnie od rozmiaru wolumenów danych, co może zostać zapewnione między innymi poprzez efektywne strategie cachowania materiałów przez system lub rozproszenie dystrybucji plików z użyciem serwerów CDN. 

\subsection{MuseScore}
MuseScore to biblioteka nutowa, która poza przeglądaniem nut pozwala odsłuchać wygenerowane nagranie utworu , które jest zsynchronizowane z podświetlaniem aktualnie odtwarzanego taktu, co pozwala na lepsze zrozumienie nut. Dodatkowo niektóre utwory zawarte w bibliotece zostały nagrane na żywo, przez co uczeń może dokładnie usłyszeć jak utwór powinien brzmieć. Aplikacja ta pozwala na tworzenie własnych zapisów nutowych oraz wprowadzenie ich do współdzielonego repozytorium wraz z opisaniem plików poprzez metadane, po których wyszukiwarka pozwala filtrować, umożliwiając szybkie znalezienie interesujących użytkownika zasobów. Dobrze zaprojektowaną skalowalność systemu można zaobserwować poprzez fakt szybkiego ładowania nut, podkładów oraz filtrowania czy wyszukiwania zasobów pomimo zarejestrowania w systemie ponad trzech i pół miliona utworów. 

MuseScore ogranicza się jednak jedynie do przechowywania zasobów nutowych, a więc użytkownicy nie mogą przechowywać w nim nagrań audio, wideo czy innych zasobów muzycznych. Dodatkowo aplikacja nie posiada wsparcia dla udostępniania zasobów jedynie określonym użytkownikom, co jest istotnym ograniczeniem w przypadku chęci udostępniania zasobów objętych prawami autorskimi, takich jak autorskie ćwiczenia do nauki, które nauczyciel może chcieć udostępnić jedynie swoim uczniom. 

\subsection{Google Drive}
Google Drive to usługa chmury udostępniająca użytkownikom możliwość przechowywania danych na serwerach Google. Rozwiązanie to wspiera wiele formatów plików, w tym formaty muzyczne, odpowiadające plikom audio, wideo, dokumentom tekstowym czy projektom programów do tworzenia muzyki. Dostęp do treści uproszczony został porpzez intuicyjny interfejs webowy oraz mobilny z szerokimi możliwościami sortowania plików, takimi jak filtrowanie po nazwie, rozmiarze, datach utworzenia czy typie. Ponad to, system indeksuje zawartość plików, umożliwiając przeszukiwanie materiałów tekstowych (PDF, TXT, DOCX i tym podobnych) według ich zawartości. 
