\chapter{Drugi rozdział}
quis dapibus.
\section{Pierwsza sekcja}
Maecenas tincidunt 
\subsection{Pierwsza podsekcja}

\begin{figure}[ht]
\centering
\includegraphics[scale=0.25]{rysunki/example}
\caption{Przykładowy obraz zamieszczony w pracy dyplomowej}
\label{img/template1}
\end{figure}
Vestibulum lorem .

\begin{table}[ht]
\captionsetup{justification=centering}
\caption{Dane techniczne silnika napędowego układu jezdnego}
\centering
    \begin{tabular}{|l|l|c|}
    \hline
    \multicolumn{1}{|l|}{\textbf{Lp.}} & \multicolumn{1}{l|}{\textbf{Parametr}} & \multicolumn{1}{c|}{\textbf{Wartość}} \\
    \hline
       1.   & Napięcie zasilania [V] & 12  \\
    \hline
       2.   &  Prędkość obrotowa [obr/min]& 200\\
    \hline
       3.  & Moment obrotowy [Nm]   & 0.8 \\
    \hline
       4.  & Maks. prąd pracy [A]      &  0.8 \\
    \hline
       5.  & średnica wału [mm]     &  8 \\
    \hline
       6.  & Rodzaj czujnika     &  Inkrementalny   \\
    \hline
       7.  & Rozdzielczość enkodera [imp/obr]   &  75 \\
    \hline
    \end{tabular}
  \label{silnik_skret}
\end{table}

Nunc egestas mauris. 

\begin{equation}
\label{rownanie}
\sum_{n=1}^{\infty} 2^{-n} \arccos(\frac{\int_{a}^{b} x^2 dx}{x})  = 1
\end{equation}

Nunc egestas m

\lstset{style=matlab}

\begin{lstlisting}[caption={Przykładowy listing programu Matlab}]
for n = 1 : 4
    
    for j = 1 : 100
        Imp_1_100(n,j) = (R1*R2*C2*i*j*2*pi + R1 + R2);
    end
    
    % Zmiana wartości  
    
    if log(abs(Imp_1_100(n,100))) - log(Rr) <= 2
        Rr = Rr/10;
    end
    
    RT(n) = Rr;  % Przypisanie wartości
end
\end{lstlisting}


\lstset{style=java}

\begin{lstlisting}[caption={Przykładowy listing w języku Java},label=javovy]
class OuterClass {
  int x = 10;

  private class InnerClass {
    int y = 5;
  }
}

public class MyMainClass {
  public static void main(String[] args) {
    OuterClass myOuter = new OuterClass();
    OuterClass.InnerClass myInner = myOuter.new InnerClass();
    System.out.println(myInner.y + myOuter.x);
  }
}
\end{lstlisting}

\newpage





\lstset{style=vhdl}

\begin{lstlisting}[caption={Przykładowy listing w języku VHDL}]

architecture FSMD of gcd is
begin

    process(rst, clk)

    -- define states using variable 
    type S_Type is (ST0, ST1, ST2);
    variable State: S_Type := ST0 ;			
    variable Data_X, Data_Y: unsigned(3 downto 0);
	
    

end FSMD;

\end{lstlisting}


Nunc egestas 



% Definition of blocks:
\tikzset{%
  block/.style    = {draw, thick, rectangle, minimum height = 3em,
    minimum width = 3em},
  sum/.style      = {draw, circle, node distance = 2cm}, % Adder
  input/.style    = {coordinate}, % Input
  output/.style   = {coordinate} % Output
}
% Defining string as labels of certain blocks.
\newcommand{\suma}{\Large$+$}
\newcommand{\inte}{$\displaystyle \int$}
\newcommand{\derv}{\huge$\frac{d}{dt}$}

\begin{figure}[ht]

\begin{tikzpicture}[auto, thick, node distance=2cm, >=triangle 45]
\draw
	% Drawing the blocks of first filter :
	node at (0,0)[right=-3mm]{\Large }
	node [input, name=input1] {} 
	node [sum, right of=input1] (suma1) {\suma}
	node [block, right of=suma1] (inte1) {\inte}
         node at (6.8,0)[block] (Q1) {\Large $Q_1$}
         node [block, below of=inte1] (ret1) {\Large$T_1$$$};
    % Joining blocks. 
    % Commands \draw with options like [->] must be written individually
	\draw[->](input1) -- node {$X(Z)$}(suma1);
 	\draw[->](suma1) -- node {} (inte1);
	\draw[->](inte1) -- node {} (Q1);
	\draw[->](ret1) -| node[near end]{} (suma1);
	% Adder
\draw
	node at (5.4,-4) [sum, name=suma2] {\suma}
    	% Second stage of filter 
	node at  (1,-6) [sum, name=suma3] {\suma}
	node [block, right of=suma3] (inte2) {\inte}
	node [sum, right of=inte2] (suma4) {\suma}
	node [block, right of=suma4] (inte3) {\inte}
	node [block, right of=inte3] (Q2) {\Large$Q_2$$$}
	node at (9,-8) [block, name=ret2] {\Large$T_2$$$}
;
	% Joining the blocks of second filter
	\draw[->] (suma3) -- node {} (inte2);
	\draw[->] (inte2) -- node {} (suma4);
	\draw[->] (suma4) -- node {} (inte3);
	\draw[->] (inte3) -- node {} (Q2);
	\draw[->] (ret2) -| (suma3);
	\draw[->] (ret2) -| (suma4);
         % Third stage of filter:
	% Defining nodes:
\draw
	node at (11.5, 0) [sum, name=suma5]{\suma}
	node [output, right of=suma5]{}
	node [block, below of=suma5] (deriv1){\derv}
	node [output, right of=suma5] (sal2){}
;
	% Joining the blocks:
	\draw[->] (suma2) -| node {}(suma3);
	\draw[->] (Q1) -- (8,0) |- node {}(ret1);
	\draw[->] (8,0) |- (suma2);
	\draw[->] (5.4,0) -- (suma2);
	\draw[->] (Q1) -- node {}(suma5);
	\draw[->] (deriv1) -- node {}(suma5);
	\draw[->] (Q2) -| node {}(deriv1);
    	\draw[<->] (ret2) -| node {}(deriv1);
    	\draw[->] (suma5) -- node {$Y(Z)$}(sal2);
    	% Drawing nodes with \textbullet
\draw
	node at (8,0) {\textbullet} 
	node at (8,-2){\textbullet}
	node at (5.4,0){\textbullet}
    	node at (5,-8){\textbullet}
    	node at (11.5,-6){\textbullet}
    	;
	% Boxing and labelling noise shapers
	\draw [color=gray,thick](-0.5,-3) rectangle (9,1);
	\node at (-0.5,1) [above=5mm, right=0mm] {\textsc{generator szumu I rzędu}};
	\draw [color=gray,thick](-0.5,-9) rectangle (12.5,-5);
	\node at (-0.5,-9) [below=5mm, right=0mm] {\textsc{generator szumu II rzędu}};
\end{tikzpicture}
\caption{Przykładowy schemat blokowy utworzony przy użyciu pakietu tikz}
\end{figure}

Nunc 


