\bibliographystyle{unsrt}                       % styl bibliografii
% \begin{thebibliography}{3}                      % początek środowiska
\addcontentsline{toc}{chapter}{Wykaz literatury}    % dodaje bibliografiê do spisu treści
\small              % spisy i bibliografie sk³adamy mniejszym stopniem pisma
% przyk³adowy wpis
% \bibliography{Praca/meta/mybib}
\bibliography{Praca/meta/mybib}
% \bibitem{Yang}
% P. Yang, H. Wang, J. Yang, Z. Qian, Y. Zhang and X. Lin, \emph{"Deep Learning Approaches for Similarity Computation: A Survey,"} in IEEE Transactions on Knowledge and Data Engineering, vol. 36, no. 12, pp. 7893-7912, Dec. 2024
%     \bibitem{Duda}      % \bibitem{etykieta}
% Duda A.: \emph{Wprowadzenie do topologii}, PWN, Warszawa 1986
% % nastêpna pozycja
%     \bibitem{EngeSiek}
% Engelking R., Sieklucki K.: \emph{Geometria i topologia. Czêœæ II. Topologia}, PWN, Warszawa 1980
% % nastêpna pozycja
%     \bibitem{Patk}
% Patkowska H.: \emph{Wstêp do topologii}, PWN, Warszawa 1979
% % nastêpna pozycja
%     \bibitem{Siek}
% Sieklucki K.: \emph{Geometria i topologia. Czêœæ I. Geometria}, PWN, Warszawa 1979
% % nastêpna pozycja
% \bibitem{link}
% National Center of Biotechnology Information, http://www.ncbi.nlm.nih.gov (data dostêpu 20.12.2012 r.).
  
% \end{thebibliography}                           % koniec œrodowiska